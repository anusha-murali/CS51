\documentclass{beamer}

% \usepackage{beamerthemesplit} // Activate for custom appearance

\title{Example Presentation Created with the Beamer Package}
\author{Till Tantau}
\date{\today}

\begin{document}

\frame
{
  \frametitle{PSET8}

\begin{enumerate}
\item Greetings!
\item This is a demo of simulating an infectious process
\item We have a simulation developed in OCaml using object oriented programming concepts
\item First I am going to show you my simulation of the infectious process with the default values in the config.ml file
\item Then I will show you how the infectious process evolves when we change various parameters in the config.ml file
\end{enumerate}
 }
 
 
 \frame
{
  \frametitle{Simulation with the default parameters}

\begin{enumerate}
\item Here I am showing the simulation with the default parameter values
\item On the LHS, we see how susceptible individuals are getting infected
\item Susceptible individuals are shown in BLUE
\item Infected individuals are shown in RED with a circle around it
\item The radius of the red circles is proportional to the required social distancing
\item The gray crosses correspond to the diseased individuals
\end{enumerate}
 }
 
  
 \frame
{
  \frametitle{Simulation with the default parameters}

\begin{enumerate}
\item On the RHS, we have a stacked bar graph, which is quite informative
\item The gray area at the bottom of the graph corresponds to the diseased people
\item The red area corresponds to the infected people
\item The blue area corresponds to the suscebtile people
\item And finally the black area at the top corresponds to the folks, who have recovered and are now immune to the disease
\item At the bottom, we have the statistics about the population
\item From the left to right, these numbers are XX of diseased, XX of infected, XX of susceptible and XX of recovered and have immunity
\end{enumerate}
 }
 
 
 
 \frame
{
  \frametitle{Changing the default parameter values}
  
\begin{enumerate}
\item Let me now repeat the simulation, after changing certain parameter values
\item I am going to update the config.ml file and compile again
\item BTW, the file, config.ml, has all the parameter values that we can play around with and observe how the changes affect the infectious process
\item I am going to change the NEIGHBOR\_RADIUS from its default value of 4 to 1. 
\item What does it mean?
\item It means that now less people will become infected. Because, the infection does not spread that far away from an infected person
\item Let me recompile and re-run the experiment...
\end{enumerate}
 
 }
 
 
  \frame
{
\frametitle{Changing the default parameter values}
  
\begin{enumerate}
\item Let me now change the NEIGHBOR\_RADIUS to 8
\item What will it do?
\item It means that the disease is more contagious - it can spread to more people than before
\item Let me recompile and re-run the experiment
\item As we can see, a lot more people got infected - and consequently, a lot more people died as a result of the infection
\end{enumerate}
  }
  
\frame
{
\frametitle{Changing the default parameter values}

\begin{enumerate}
\item Now, let me increase the IMMUNITY PERIOD
item First, let me change the NEIGHBOR\_RADIUS to 4
\item The default immunity duration is 100 time steps
\item Let me change it to 200 time steps
\item This means a person who was previously infected and recovered will have twice as long as immune to the disease
\item Let me recompile and re-run. 
\item Here it is ... As we can see, more people have recovered and are immune to the disease. Because the black area is very large
\end{enumerate}
}


\frame
{
\frametitle{Changing the default parameter values}

\begin{enumerate}
\item There are many more parameters that we can change and see how the infectious process are impacted by those changes
\item For  example, the probability of mortality or mortality rate
\item The default value in the config.ml file is 0.02
\item However, for certain infectious diseases this can be higher
\item For both the IMMUNITY PERIOD and RECOVERY PERIOD, not only can we change their mean values, but also the standard deviation - so this simulation can very closely model a true infectious process
\item Thank you so much!
\end{enumerate}



}
 
\end{document}

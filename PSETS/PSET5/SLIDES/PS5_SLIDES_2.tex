\documentclass{beamer}
%\documentclass[xcolor={dvipsnames,table}]{beamer}
%\usetheme{Warsaw}
%\usetheme{AnnArbor}
%\usetheme{Frankfurt}
\usetheme{Madrid}
%\setbeamertemplate{title page}[default][colsep=-0bp,rounded=false]
%\setbeamertemplate{frametitle}[default][colsep=-4bp,rounded=false,shadow=false]
%\setbeamertemplate{blocks}[rounded][shadow=false]


\title{PSET5: Ordered Collections \\ Priority Queues}


\date{March 17, 2023}

\author{Anusha Murali}


\usepackage{tikz-qtree}
\begin{document}

%\frame{\titlepage}

\begin{frame}[fragile]
\titlepage

\end{frame}


\begin{frame}[fragile]
\frametitle{Tasks to do}

\begin{block}{Part II: Implement Ordered Collections with Priority Queues}
\begin{enumerate}
\item Complete \textcolor{red}{ListQueue}: elements are stored a list
\item Complete \textcolor{red}{TreeQueue}: elements are stored in a BST
\item Complete \textcolor{red}{BinaryHeap}: elements are stored in a balanced binary tree
\end{enumerate}
\end{block}
\end{frame}


\begin{frame}[fragile]

\vspace*{0.5in}

\centerline{\huge ListQueue}

\vspace*{0.2in}
\begin{center}
Task: Use a a simple list to store the queue elements in \textcolor{red}{sorted order}.
\end{center}

\end{frame}





%%%%%%%%%%%%%%%%%%%%%%%
\begin{frame}[fragile]
\frametitle{Step 1 Load Files}

\begin{block}{Load Files}
\begin{enumerate}
\item \# \#use "order.ml"
\item \# \#use "orderedcoll.ml"
\item \# \#use "prioqueue.ml"
\end{enumerate}
\end{block}
\end{frame}


\begin{frame}{fragile}
\frametitle{Step 2: Complete the {\tt ListQueue} functor}

\begin{alertblock}{Complete empty}
\hspace*{1.0in}let empty : queue = \\
\hspace*{1.15in}      failwith "ListQueue empty not implemented"
\end{alertblock}

\begin{block}{empty}
\hspace*{1.0in}let empty : queue = []
\end{block}


\begin{alertblock}{Complete is\_empty}
\hspace*{1.0in}let is\_empty (q : queue) : bool = \\
\hspace*{1.15in}      failwith "ListQueue is\_empty not implemented"
\end{alertblock}

\begin{block}{is\_empty}
\hspace*{1.0in}let is\_empty (q : queue) : bool = (q = empty)
\end{block}

\end{frame}



\frame[fragile]
\frametitle{Step 2: Complete the {\tt ListQueue} functor}

\begin{alertblock}{Complete add}
\hspace*{1in}let add (e : elt) (q : queue) : queue = \\
\hspace*{1.15in}      failwith "ListQueue add not implemented"
\end{alertblock}

\begin{block}{add}
\begin{verbatim}
let rec add (e : elt) (q : queue) : queue =
      match q with
      | [] -> e :: []
      | hd :: tl -> (match Elt.compare e hd with
                     | Less -> e :: q
                     | Greater -> hd :: (add e tl)
                     | Equal -> hd :: (add e tl)
                     )
\end{verbatim}
\end{block}
\end{frame}




      
      
\frame[fragile]
\frametitle{Step 2: Complete the {\tt ListQueue} functor}

\begin{alertblock}{Complete take}
\hspace*{1in}let take (q : queue) : elt * queue = \\
\hspace*{1.15in}      failwith "ListQueue take not implemented"
\end{alertblock}

\begin{block}{take}
\begin{verbatim}
let take (q : queue) : elt * queue =
      match q with
      | [] -> raise QueueEmpty
      | hd :: tl -> (hd, tl)
\end{verbatim}
\end{block}
\end{frame}
      




\begin{frame}{fragile}
\frametitle{Step 2: Complete the {\tt ListQueue} functor}

\begin{block}{Re-load prioqueue.ml (with your completed functions)}
\begin{enumerate}
\item \# \#use "prioqueue.ml"
\end{enumerate}
\end{block}

\begin{example}
Now the examples in the next slides should work
\end{example}
\end{frame}
      
      
\begin{frame}{fragile}
\frametitle{Test all the new functions}

\begin{example}
\begin{enumerate}
\item First create an empty queue called myQ \\
        \# {\bf  let myQ  = IntListQueue.empty;;}
\item Check if myQ is empty or not \\
        \# {\bf IntListQueue.is\_empty myQ}
\item Insert 65\\
 \# {\bf let myQ = IntListQueue.add 65 myQ};;
 \item Print myQ\\
\# {\bf IntListQueue.to\_string myQ;;}
\item Insert 7\\
 \# {\bf let myQ = IntListQueue.add 7 myQ};;
 \item Print myQ\\
\# {\bf IntListQueue.to\_string myQ;;}
\end{enumerate}
\end{example}

\end{frame}


\begin{frame}{fragile}
\frametitle{Test all the new functions}

\begin{example}
\begin{enumerate}
\item Check if myQ is empty or not \\
        \# {\bf IntListQueue.is\_empty myQ}
\item Insert 29. It should be inserted between 7 and 65\\
 \# {\bf let myQ = IntListQueue.add 29 myQ};;
 \item Print myQ\\
\# {\bf IntListQueue.to\_string myQ;;}
\item Try the take function. It should return the first element (which is 7 in this case, and it has the highest priority) and the rest of the list. \\
 \# {\bf let (x, myQ) = IntListQueue.take myQ};;
 \item Print myQ. It should only have [29;65] now.\\
\# {\bf IntListQueue.to\_string myQ;;}
\end{enumerate}
\end{example}
\end{frame}


\begin{frame}

\centerline{\huge Test using IntString}

\end{frame}








\begin{frame}{fragile}
\frametitle{Test all the new functions}

\begin{block}{First create the \textcolor{yellow}{IntStringListQueue} module}

\vspace*{0.1in}

\noindent
        \# {\bf module IntStringListQueue = (ListQueue(IntStringCompare) : \\
 \hspace*{0.5in}  PRIOQUEUE with type elt = IntStringCompare.t);;}

\end{block}

\end{frame}


\begin{frame}{fragile}
\frametitle{Test all the new functions}

\begin{example}
\begin{enumerate}
\item First create an empty queue called myQ \\
        \# {\bf  let myQ  = IntStringListQueue.empty;;}
\item Check if myQ is empty or not \\
        \# {\bf IntStringListQueue.is\_empty myQ}
\item Insert (65, "Anusha")\\
 \# {\bf let myQ = IntStringListQueue.add (65, "Anusha") myQ;; }
 \item Print myQ\\
\# {\bf IntStringListQueue.to\_string myQ;;}
\item Insert (7, "CS51")\\
 \# {\bf let myQ = IntStringListQueue.add (7, "CS51") myQ};;
 \item Print myQ\\
\# {\bf IntStringListQueue.to\_string myQ;;}
\end{enumerate}
\end{example}

\end{frame}




\begin{frame}{fragile}
\frametitle{Test all the new functions}

\begin{example}
\begin{enumerate}
\item Check if myQ is empty or not \\
        \# {\bf IntStringListQueue.is\_empty myQ}
\item Insert (29, "PHYS143"). It should be inserted between (7, "CS51") and (65, "Anusha")\\
 \# {\bf let myQ = IntStringListQueue.add (29, "PHYS143") myQ};;
 \item Print myQ\\
\# {\bf IntStringListQueue.to\_string myQ;;}
\item Try the take function. It should return the first element (which is (7, "CS51"), and it has the highest priority) and the rest of the list. \\
 \# {\bf let (x, myQ) = IntStringListQueue.take myQ};;
 \item Print myQ. It should only have [(29, "PHYS143"); (65, "Anusha")] now.\\
\# {\bf IntStringListQueue.to\_string myQ;;}
\end{enumerate}
\end{example}
\end{frame}





\begin{frame}{fragile}
\frametitle{Important comments on ListQueue}
\begin{block}{Average and worst-case time complexity}
\begin{enumerate}
\item On the average, the element is found in the middle of the list. So, we need to search $n/2$ items. So the average time complexity is $O(n)$
\item In the worst case, the element is found at the end of the list. So, the worst-case time complexity is also $O(n)$
\end{enumerate}
\end{block}

\end{frame}


\end{document}

